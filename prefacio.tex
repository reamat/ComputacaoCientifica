%Este trabalho está licenciado sob a Licença Creative Commons Atribuição-CompartilhaIgual 3.0 Não Adaptada. Para ver uma cópia desta licença, visite https://creativecommons.org/licenses/by-sa/3.0/ ou envie uma carta para Creative Commons, PO Box 1866, Mountain View, CA 94042, USA.

\chapter*{Prefácio}
\addcontentsline{toc}{chapter}{Prefácio}

Este livro busca introduzir a linguagem de programação C no contexto de computação científica. Alguns livros de programação em linguagem C, tais como \cite{KERNIGHAM,SENNE,DAMAS,SCHILDT}, estão focados no básico da linguagem, outros, tais como \cite{CALSCI,BURDEN}, estão focados nos métodos numéricos. O excelente livro \cite{PRESSC} faz as duas coisas, no entanto, ele não apresenta as bibliotecas para computação científica. Aqui, o interesse é trabalhar a linguagem de programação C, dando foco aos itens de interesse para resolver problemas em computação científica, entre eles as bibliotecas GSL e LAPACK.