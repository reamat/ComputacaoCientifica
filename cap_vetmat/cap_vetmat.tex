
\chapter{Vetores e matrizes}
\section{Vetores}
Vetor (ou array) é uma lista de elementos do mesmo tipo que podem ser acessados individualmente com o mesmo nome de variável. Definimos um vetor dizendo o tipo, o nome da variável e o número de elementos: \verb|tipo nome[nº elementos];|.
\begin{ex}\label{ex15.0}Alguns testes de inicialização de vetores.
\end{ex}
\begin{verbatim}
#include <stdio.h>
#include <math.h>

main (void)
{
  int i;
  double x[3]={1.5,2,3};
  int y[]={1,2,5};
  int z[8]={1,2,5};

  for (i=0;i<3;i++)     printf("x[%d] = %f\n",i,x[i]);
  printf("\n");
  for (i=0;i<3;i++)     printf("y[%d] = %d\n",i,y[i]);
  printf("\n");
  for (i=0;i<8;i++)     printf("z[%d] = %d\n",i,z[i]);
} 
\end{verbatim}


\begin{ex}\label{ex15}Implemente um programa que armazena em um vetor os 10 primeiros números da sequência de Fibonacci e imprime a soma do sétimo com o décimo.
\end{ex}
\begin{verbatim}
 #include <stdio.h>

main (void)
{
  int i,x[10];
  x[0]=1;
  x[1]=1;
  for (i=0;i<8;i++)
  {
    x[i+2]=x[i+1]+x[i];
  }
  printf("soma do sétimo com o décimo = %d\n",x[6]+x[9]);
}
\end{verbatim}
No exemplo \ref{ex15}, definimos um vetor com 10 posições fazendo \verb|int x[10];|. Depois, acessamos a posição \verb|i| do vetor fazendo \verb|x[i]|. Observe que as posições do vetor começamos em \verb|0| e, portanto, a sétima posição é acessada por \verb|x[6]| e a décima por \verb|x[9]|.

\begin{ex}\label{ex_Jacobi}O método iterativo de Jacobi para calcular a solução do sistema linear
\begin{eqnarray}
\nonumber a_{11}x_1+a_{12}x_2+\cdots+a_{1n}x_n&=&y_1\\
\nonumber a_{21}x_1+a_{22}x_2+\cdots+a_{2n}x_n&=&y_2\\
\nonumber &\vdots&     \\
\label{sistema} a_{n1}x_1+a_{n2}x_2+\cdots+a_{nn}x_n&=& y_n
\end{eqnarray}
é dado pela recursão
\begin{eqnarray*}
x_1^{(k+1)}&=&\frac{y_1 - \left(a_{12}x_2^{(k)}+\cdots+a_{1n}x_n^{(k)}\right)}{a_{11}}\\
x_2^{(k+1)}&=&\frac{y_2 - \left(a_{21}x_1^{(k)}+a_{23}x_3^{(k)}+\cdots+a_{2n}x_n^{(k)}\right)}{a_{22}}\\
&\vdots&\\
x_n^{(k+1)}&=&\frac{y_2 - \left(a_{n1}x_1^{(k)}+\cdots+a_{n,n-2}x_{n-2}^{(k)}+a_{n,n-1}x_{n-1}^{(k)}\right)}{a_{nn}}
\end{eqnarray*}
onde $x_1^0$, $x_2^0$, $\cdots$, $x_n^0$ é um chute inicial. A iteração converge para a solução quando a matriz do sistema é estritamente diagonal dominante.

Implemente um código para resolver o sistema
\begin{eqnarray*}
3x+y+z&=&1\\
-x-4y+2z&=&2\\
-2x+2y+5z&=&3
\end{eqnarray*}
com o método de Jacobi.
\end{ex}
Primeiro, montamos a iteração:
\begin{eqnarray*}
x^{k+1}&=&\frac{1-y^{k}-z^{k}}{3}\\
y^{k+1}&=&\frac{-2-x^{k}+2z^{k}}{4}\\
z^{k+1}&=&\frac{3+2x^{k}-2y^{k}}{5}
\end{eqnarray*}
Vamos usar o chute inicial $x^0=0$, $y^0=0$ e $z^0=0$.
\begin{verbatim}
#include <stdio.h>
#include <math.h>

main (void)
{
  double norma2,tolerancia=1e-3,x_antigo[3],x_atual[3];
  int i,controle=3;
  //chute inicial
  for (i=0;i<3;i++)x_antigo[i]=0;

  //iteração
  while (controle)
  {
    x_atual[0]=(1-x_antigo[1]-x_antigo[2])/3;
    x_atual[1]=(-2-x_antigo[0]+2*x_antigo[2])/4;
    x_atual[2]=(3+2*x_antigo[0]-2*x_antigo[1])/5;
    norma2=0;
    for (i=0;i<3;i++) norma2+=(x_atual[i]-x_antigo[i])*(x_atual[i]-x_antigo[i]);
    if (norma2<tolerancia) controle--;
    else controle=3;
    for (i=0;i<3;i++) x_antigo[i]=x_atual[i];
    printf("x=%f, y=%f, z=%f\n",x_atual[0],x_atual[1],x_atual[2]);	
  }
}
\end{verbatim}
Nessa versão de código usamos o critério de parada norma em $l_2$ da diferença menor que uma tolerância, isto é,
$$
||(x^{k+1}-x^{k},y^{k+1}-y^{k},z^{k+1}-z^{k})||_2=\sqrt{(x^{k+1}-x^{k})^2+(y^{k+1}-y^{k})^2+(z^{k+1}-z^{k})^2}<TOL.
$$
Também usamos o dispositivo de controle, isto é, depois que atingimos a tolerança caminhamos mais três passos.

Agora vamos fazer um vesão um pouco mais elaborada. Vamos criar uma função que calcula norma $l_2$ de vetores e outra que faz uma iteração. Para isso, temos que passar um vetor como parâmetro para a função. Observe o código:
\begin{verbatim}
#include <stdio.h>
#include <math.h>

double norma_dif_2(double x[3],double y[3])
{
  int i;
  double norma2=0;
  for (i=0;i<3;i++) norma2+=(x[i]-y[i])*(x[i]-y[i]); 
  return sqrt(norma2);	
}
void Iteracao(double x_antigo[3],double x_atual[3])
{
    x_atual[0]=(1-x_antigo[1]-x_antigo[2])/3.;
    x_atual[1]=(-2-x_antigo[0]+2*x_antigo[2])/4.;
    x_atual[2]=(3+2*x_antigo[0]-2*x_antigo[1])/5.;
}
void Jacobi(double tolerancia,double x_atual[3])
{
  double x_antigo[3];
  int i,controle=3;
  //chute inicial
  for (i=0;i<3;i++)x_antigo[i]=0;

  //iteração
  while (controle)
  {
    Iteracao(x_antigo,x_atual);
    if (norma_dif_2(x_antigo,x_atual)<tolerancia) controle--;
    else controle=3;
    for (i=0;i<3;i++) x_antigo[i]=x_atual[i];
    //printf("x=%f, y=%f, z=%f\n",x_atual[0],x_atual[1],x_atual[2]);	
  }
  return;
}

main (void)
{
  int i;
  double tolerancia=1e-3,solucao[3];
  Jacobi(tolerancia,solucao);
  for (i=0;i<3;i++)     printf("x[%d] = %f\n",i,solucao[i]);
}
\end{verbatim}
Esse último código passa vetores como parâmetros para funções. A função \verb|norma_dif_2| passa dois vetores e retorna \verb|double|. A \verb|Iteracao| passa dois vetores e retorna \verb|void|, onde usamos um vetor para levar a informação para dentro da função e o outro vetor para retirar o resultado: \verb|x_antigo| leva a informação e \verb|x_atual| entra com qualquer coisa e é calculado dentro da função. Já a função \verb|Jacobi| entra o valor da tolerância e passa um vetor para colocar a solução.

Os códigos que implementamos resolve apenas um único sistema $3$x$3$. É interessante pensar em rotinas mais gerais, que resolvem um sistema $Ax=b$ de dimensão $n$x$n$.


\section{Matrizes}
Matrizes são vetores de vetores, isto é, cada posição de um vetor pode ser um vetor. Para definir uma matriz fazemos \verb|tipo nome[nº linhas][nº colunas]|, como no exemplo:
\begin{ex}\label{ex16}Alguns testes de inicialização de matrizes.
\end{ex}
\begin{verbatim}
#include <stdio.h>
#include <math.h>

main (void)
{
  int i,j;
  double x[2][2];// não inicializado
  int y[3][3]={{1,2,5},{1,0,1},{0,1,0}};
  x[0][0]=1.1;
  x[0][1]=2;
  x[1][0]=3;
  x[1][1]=4;
  for (i=0;i<2;i++) 
  {  
    for (j=0;j<2;j++) 
    {
    printf("x[%d][%d] = %f    ",i,j,x[i][j]);
    }
  printf("\n");
  }
  
  printf("\ny[%d][%d] = %d\n",1,2,y[1][2]);
}
\end{verbatim}

\begin{ex}\label{ex_Jacobi_1}Vamos refazer o exemplo \ref{ex_Jacobi} usando estrutura de matrizes
\end{ex}
\begin{verbatim}
#include <stdio.h>
#include <math.h>

main (void)
{
  double vetor[3]={1,2,3},matriz[3][3]={{3,1,1},{-1,-4,2},{-2,2,5}};
  double norma2,tolerancia=1e-3,x_antigo[3],x_atual[3];
  int i,controle=3;
  //chute inicial
  for (i=0;i<3;i++)x_antigo[i]=0;

  //iteração
  while (controle)
  {
    x_atual[0]=(vetor[0]-matriz[0][1]*x_antigo[1]-matriz[0][2]*x_antigo[2])/matriz[0][0];
    x_atual[1]=(vetor[1]-matriz[1][0]*x_antigo[0]-matriz[1][2]*x_antigo[2])/matriz[1][1];
    x_atual[2]=(vetor[2]-matriz[2][0]*x_antigo[0]-matriz[2][1]*x_antigo[1])/matriz[2][2];
    norma2=0;
    for (i=0;i<3;i++) norma2+=(x_atual[i]-x_antigo[i])*(x_atual[i]-x_antigo[i]);
    if (norma2<tolerancia) controle--;
    else controle=3;
    for (i=0;i<3;i++) x_antigo[i]=x_atual[i];
    printf("x=%f, y=%f, z=%f\n",x_atual[0],x_atual[1],x_atual[2]);	
  }
}
\end{verbatim}
Agora, vamos fazer o mesmo código outra vez generalizando as rotinas
\begin{verbatim}
#include <stdio.h>
#include <math.h>

#define N 3 /* dimensão do sistema*/

double norma_2(double x[N])
{
  int i;
  for (i=0;i<N;i++)
  {
    norma2+=x[i]*x[i];
  }
  return sqrt(norma2);	
}
void Iteracao(double x_antigo[N],double x_atual[N],double matriz[N][N], double vetor[N])
{
  double aux;
  int i,j;
  for (i=0;i<N;i++)
  {
    aux=0;
    for (j=0;j<i;j++)
    aux+=matriz[i][j]*x_antigo[j];
    for (j=i+1;j<N;j++)
    aux+=matriz[i][j]*x_antigo[j];
  x_atual[i]=(vetor[i]-aux)/matriz[i][i];
  }
}
void Jacobi(double tolerancia,double x_atual[N])
{
  double x_antigo[N],dif[N];
  double matriz[N][N]={{3,1,1},{-1,-4,2},{-2,2,5}};
  double vetor[N]={1,2,3};
  int i,controle=3;
  //chute inicial
  for (i=0;i<N;i++)x_antigo[i]=0;

  //iteração
  while (controle)
  {
    Iteracao(x_antigo,x_atual,matriz,vetor);
    for (i=0;i<N;i++) dif[i]=x_atual[i]-x_antigo[i];
    if (norma_2(dif)<tolerancia) controle--;
    else controle=3;
    for (i=0;i<N;i++) x_antigo[i]=x_atual[i];
  }
  return;
}

main (void)
{
  int i;
  double tolerancia=1e-3,solucao[N];
  Jacobi(tolerancia,solucao);
  for (i=0;i<N;i++)     printf("x[%d] = %f\n",i,solucao[i]);
}
\end{verbatim}
No último código usamos uma diretiva \verb|#define N 3| que diz para o compilador trocar todos os $N$ do código por $3$. Observe que a constante $N$ é bastante usada no código e a diretiva \verb|# define| desobriga de passá-la como parâmetro em todas as funções que ela aparece.



\section{Problemas}
\begin{ex}\label{ex_Jacobi_2} Resolva numericamente o problema de valor de contorno
  \begin{eqnarray*}
    -u_{xx} &=& 100(x-1)^2,\quad 0 < x < 1,\label{eq:pvc2-eq}\\
    u(0) &=& 0,\label{eq:pvc2-bc1}\\
    u(1) &=& 0.\label{eq:pvc2-bc2}
  \end{eqnarray*}
usando a fórmula de diferenças finitas central de ordem 2 para discretizar a derivada. Compute o erro absoluto médio definido por
\begin{equation*}
  E := \frac{1}{N}\sum_{i=1}^N \left|u(x_i) - u_i\right|,
\end{equation*}
onde $x_i$ é o $i$-ésimo ponto da malha, $i=1, 2, \dotsc, N$ e $N$ é o número de pontos da mesma, $u_i$ é a solução aproximada e $u(x_i)$ é a solução analítica, dada por
\begin{equation*}
  u(x) = -\frac{25(x-1)^4}{3} - \frac{25x}{3} + \frac{25}{3}.
\end{equation*}
\end{ex}
Solução: Primeiro, vamos montar a discretização do problema que está definido no domínio $[0,1]$. A malha é dada por:
  \begin{equation*}
    x_i = (i-1)h,\quad i=1, 2, \dotsc, N,
  \end{equation*}
com $h = 1/(N-1)$.

Aplicando na equação diferencial o método de diferenças finitas central de ordem 2 para aproximar a derivada segunda, obtemos as seguintes $N-2$ equações:
\begin{equation*}
  - \frac{u_{i-1} - 2u_i + u_{i+1}}{h^2} = 100(x_i - 1)^2,\quad i = 2, \dotsc, N-1.
\end{equation*}
As equações de contorno levam as condições:
\begin{equation*}
  u_1 = u_N = 0.
\end{equation*}
Ou seja, obtemos o seguinte sistema linear $N\times N$:
\begin{eqnarray}
  u_1 &=& 0,\label{eq:pvc2_disc_bc1}\\
  -\frac{1}{h^2}\left(u_{i-1} - 2u_i + u_{i+1}\right) &=& 100(x_i-1)^2,\quad i=2, \dotsc, N-1,\\
  u_N &=& 0.\label{eq:pvc2_disc_bc2}
\end{eqnarray}
Observamos que este é um sistema linear $N\times N$, o qual pode ser escrito na forma matricial $A\underline{u} = b$, cujos matriz de coeficientes é
\begin{equation*}
  A = 
  \begin{bmatrix}
    1 & 0 & 0 & 0 & 0 & \cdots & 0\\
    1 & -2 & 1 & 0 & 0 & \cdots & 0\\
    0 & 1 & -2 & 1 & 0 & \cdots & 0\\
    \vdots & \vdots & \vdots & \vdots & \vdots & \vdots & \vdots\\
    0 & 0 & 0 & 0 & 0 & \cdots & 1
  \end{bmatrix},
\end{equation*}
o vetor das incógnitas e o vetor dos termos constantes são
\begin{equation*}
  \underline{u} =
  \begin{bmatrix}
    u_1\\
    u_2\\
    u_3\\
    \vdots\\
    u_N
  \end{bmatrix}\quad\text{e}\quad
  b =
  \begin{bmatrix}
    0\\
    -100h^{2}(x_2-1)^2\\
    -100h^{2}(x_3-1)^2\\
    \vdots\\
    0
  \end{bmatrix}.
\end{equation*}
Vamos aplicar o método de Jacobi para resolver esse problema. Partimos do exemplo \ref{ex_Jacobi_1} e introduzimos a matriz oriunda da discretização:
\begin{verbatim}
#include <stdio.h>
#include <math.h>

#define N 101 /* dimensão do sistema*/

double norma_2(double x[N])
{
  int i;
  double norma2=x[0]*x[0]/(2*(N-1));
  for (i=1;i<N-1;i++)
  {
    norma2+=x[i]*x[i]/(N-1);
  }
  norma2+=x[N-1]*x[N-1]/(2*(N-1));
  return sqrt(norma2);	
}
void Iteracao(double u_antigo[N],double u_atual[N],double matriz[N][N], double vetor[N])
{
  double aux;
  int i,j;
  for (i=0;i<N;i++)
  {
    aux=0;
    for (j=0;j<i;j++)
    aux+=matriz[i][j]*u_antigo[j];
    for (j=i+1;j<N;j++)
    aux+=matriz[i][j]*u_antigo[j];
    u_atual[i]=(vetor[i]-aux)/matriz[i][i];
  }
}
void Jacobi(double tolerancia,double u_atual[N],double vetor[N],double matriz[N][N],double malha[N])
{

  double u_antigo[N],dif[N];
  int i,controle=3;

  //chute inicial
  for (i=0;i<N;i++)u_antigo[i]=0;

  //iteração
  while (controle)
  {
    Iteracao(u_antigo,u_atual,matriz,vetor);
    for (i=0;i<N;i++) dif[i]=u_atual[i]-u_antigo[i];
    if (norma_2(dif)<tolerancia) controle--;
    else controle=3;
    for (i=0;i<N;i++) u_antigo[i]=u_atual[i];
  }
  return;
}

double solucao_analitica(double x)
{
return -25.*(x-1.)*(x-1.)*(x-1.)*(x-1.)/3-25.*x/3+25./3.;
}

main (void)
{
  int i,j;
  double tolerancia=1e-5,solucao[N];
  double matriz[N][N];
  double vetor[N],x[N];
  //malha
  double h=1./(N-1); 
  for (i=0;i<N;i++) x[i]=i*h;
  
  
  //matriz e vetor
  for (i=0;i<N;i++) for (j=0;j<N;j++) matriz[i][j]=0;
  matriz[0][0]=1;
  vetor[0]=0;
  for (i=1;i<N-1;i++) 
  {
    matriz[i][i]=-2;
    matriz[i][i+1]=1;
    matriz[i][i-1]=1;
    vetor[i]=-100*h*h*(x[i]-1)*(x[i]-1);
  }
  matriz[N-1][N-1]=1;
  vetor[N-1]=0;

  Jacobi(tolerancia,solucao,vetor,matriz,x);
  //erro médio
  double erro=0;
  for (i=0;i<N;i++) erro+=fabs(solucao[i]-solucao_analitica(x[i]));
  erro/=N;
  printf("erro médio = %f\n",erro);
  //for (i=0;i<N;i++)     printf("u_%d=%f, u(x[%d]) = %f\n",i,solucao[i],i,
  solucao_analitica(x[i]));
}
\end{verbatim}


\section{Exercícios}
\begin{exer}\label{GS_exerc}
Implemente um código para resolver o sistema
\begin{eqnarray*}
5x_1+x_2+x_3-x_4&=&1\\
-x_1-4x_2+2x_4&=&-2\\
-2x_1+2x_2-5x_3-x_4&=&-5\\
x_2-x_3+5x_4&=&-5\\
\end{eqnarray*}
com o método de Jacobi.
\end{exer}

\begin{exer}
Implemente um código para resolver o sistema
\begin{eqnarray*}
5x_1+x_2+x_3-x_4&=&1\\
-x_1-4x_2+2x_4&=&-2\\
-2x_1+2x_2-5x_3-x_4&=&-5\\
x_2-x_3+5x_4&=&-5\\
\end{eqnarray*}
com o método de Gauss-Seidel.

A iteração de Gauss-Seidel para resolver o sistema \ref{sistema} é
\begin{eqnarray*}
x_1^{(k+1)}&=&\frac{y_1 - \left(a_{12}x_2^{(k)}+\cdots+a_{1n}x_n^{(k)}\right)}{a_{11}}\\
x_2^{(k+1)}&=&\frac{y_2 - \left(a_{21}x_1^{(k+1)}+a_{23}x_3^{(k)}+\cdots+a_{2n}x_n^{(k)}\right)}{a_{22}}\\
&\vdots&\\
x_n^{(k+1)}&=&\frac{y_2 - \left(a_{n1}x_1^{(k+1)}+\cdots+a_{n(n-1)}x_{n-1}^{(k+1)}\right)}{a_{nn}}
\end{eqnarray*}
onde $x_1^0$, $x_2^0$, $\cdots$, $x_n^0$ é um chute inicial.
\end{exer}


\begin{exer}\label{TDMA0}Um sistema tridiagonal é um sistema de equações lineares cuja matriz associada é tridiagonal, conforme a seguir:
\begin{eqnarray*}\label{linsis:EqTriDiag} \begin{bmatrix}
   {b_1} & {c_1} & {   } & {   } & {   } \\
   {a_2} & {b_2} & {c_2} & {   } & {   } \\
   {   } & {a_3} & {b_3} & \ddots & {   } \\
   {   } & {   } & \ddots & \ddots & {c_{n-1}}\\
   {   } & {   } & {   } & {a_n} & {b_n}\\
\end{bmatrix}
\begin{bmatrix}
   {x_1 }  \\
   {x_2 }  \\
   {x_3 }  \\
   \vdots   \\
   {x_n }  \\
\end{bmatrix}
=
\begin{bmatrix}
   {d_1 }  \\
   {d_2 }  \\
   {d_3 }  \\
   \vdots   \\
   {d_n }  \\
\end{bmatrix}
.
\end{eqnarray*}


A solução do sistema tridiagonal acima pode ser obtido pelo algoritmo de Thomas dado por:
\begin{eqnarray*}\label{linsis:TriDiag_1}
 c'_i &=&
\begin{cases}
\begin{array}{lcl}
  \frac{c_i}{b_i}                   , & i = 1 \\
  \frac{c_i}{b_i - a_i c'_{i - 1}}  , & i = 2, 3, \dots, n-1
\end{array}
\end{cases}\\
\text{e}\nonumber\\
d'_i &=&
\begin{cases}
\begin{array}{lcl}
  \frac{d_i}{b_i}                   , & i = 1 \\
  \frac{d_i - a_i d'_{i - 1}}{b_i - a_i c'_{i - 1}}  , & i = 2, 3, \dots, n.
\end{array}
\end{cases}
\end{eqnarray*}
Finalmente a solução final é obtida por substituição reversa:
\begin{eqnarray*}\label{linsis:TriDiag_2}
x_n &=& d'_n\\
 x_i &=& d'_i - c'_i x_{i + 1}, \qquad   i = n - 1, n - 2, \ldots, 1.
\end{eqnarray*}

Implemente um código para resolver sistemas tridiagonais e teste resolvendo o problema
$$ \begin{bmatrix}
   { 2 } & { 1 } & { 0 } & { 0 } & { 0 } \\
   { 1 } & { 2 } & { 1 } & { 0 } & { 0 } \\
   { 0 } & { 1 } & { 2 } & { 1 } & { 0 } \\
   { 0 } & { 0 } & { 1 } & { 2 } & { 1 } \\
   { 0 } & { 0 } & { 0 } & { 1 } & { 2 }
 \end{bmatrix}
\begin{bmatrix}
   {x_1 }  \\
   {x_2 }  \\
   {x_3 }  \\
   {x_4 }  \\
   {x_5 }
\end{bmatrix}
=
\begin{bmatrix}
   {4}  \\
   {4 }  \\
   {0}  \\
   {0}\\
   {2 }
\end{bmatrix}
.
$$
\end{exer}


\begin{exer}Refaça o exercício \ref{GS_exerc} usando a estrutura de matrizes.
 
\end{exer}

\begin{exer}Escreva um programa que lê as entradas de uma matriz diagonal estritamente dominante $A$ e um vetor $B$ e imprime a solução do sistema $Ax=B$. Use o método iterativo de Gauss-Seidel.
\end{exer}


\begin{exer}\label{exerc_8.1}Resolva numericamente o problema de valor de contorno para a equação de calor no estado estacionário
$$\left\{\begin{array}{l}-u_{xx}=32,~~ 0<x<1.\\
u(0)=5\\
u(1)=10\end{array}
\right.
$$
usando o método de diferenças finitas.
\end{exer}

\begin{exer}\label{exerc_8.2}Resolva numericamente o problema de valor de contorno para a equação de calor no estado estacionário
$$\left\{\begin{array}{l}-u_{xx}=200e^{-(x-1)^2},~~ 0<x<2.\\
u(0)=120\\
u(2)=100\end{array}
\right.
$$
usando o método de diferenças finitas.
\end{exer}

\begin{exer}\label{exerc_8.3}Resolva numericamente o problema de valor de contorno para a equação de calor no estado estacionário
$$\left\{\begin{array}{l}-u_{xx}=200e^{-(x-1)^2},~~ 0<x<2.\\
u'(0)=0\\
u(2)=100\end{array}
\right.
$$
usando o método de diferenças finitas.
\end{exer}
