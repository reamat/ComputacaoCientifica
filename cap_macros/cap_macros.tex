%Este trabalho está licenciado sob a Licença Creative Commons Atribuição-CompartilhaIgual 3.0 Não Adaptada. Para ver uma cópia desta licença, visite https://creativecommons.org/licenses/by-sa/3.0/ ou envie uma carta para Creative Commons, PO Box 1866, Mountain View, CA 94042, USA.

\chapter{Macros e pré-processamento}
\section{Macros}
São diretivas do pré-processamento que permitem substituir porções de código antes da compilação. As instruções do pré-processamento são seguidas do operador \verb|#|. 
\begin{itemize}
 \item[a)] \verb|#define cod exp| substitui o \verb|cod| por \verb|verb|.
 \item[b)] \verb|#include <cod.h>| inclui o código da biblioteca \verb|cod.h|.  Essa sintaxe vale para as biliotecas padrão.
 \item[c)] \verb|#include "home/fulano/aula/cod.h"| inclui o código da biblioteca \verb|cod.h|. Essa sintaxe vale para bibliotecas próprias salva com o nome \verb|cod.h| na pasta \verb|home/fulano/aula/|.
\end{itemize}
\begin{ex}
Implemente uma biblioteca com uma função que troca \verb|x| e \verb|y|. 
\end{ex}
Salvamos um arquivo com nome \verb|lib.h| com as seguintes linhas:
\begin{verbatim}
//lib.h
void troca(double *a,double *b)
{
double aux;
aux=*a;
*a=*b;
*b=aux;
}
\end{verbatim}
Depois, salvamos um arquivo com o nome teste.c com as linhas:
\begin{verbatim}
#include <stdio.h>
#include <stdlib.h>

#include "lib.h"

#define Mult(x,y) (x)*(y)

void main (int argc,char **argv)
{
  if (argc!=3)
  {
    printf("Digite dois números\n");
    return;
  }
  double x=atof(argv[1]);
  double y=atof(argv[2]);
  printf("x=%f, y=%f,  xy=%f\n",x,y,Mult(x,y));
  troca(&x,&y);
  printf("x=%f, y=%f,  xy=%f\n",x,y,Mult(x,y));
}
\end{verbatim}

\begin{ex}
Várias implementações da função mínimos entre dois números.
\end{ex}
Primeiro implementamos um a biblioteca com três tipos de função mínimo:
\begin{verbatim}
//salva como lib.h
double Min1(double a,double b)
{
if (a<b) return a;
else return b;
}

void Min2(double *a,double b)
{
if (b<*a) *a=b;
}

double Min3(double a,double b)
{
return a<b?a:b;
}
\end{verbatim}
Depois, implementamos um código que inclui a primeira biblioteca:
\begin{verbatim}
#include <stdio.h>
#include <stdlib.h>

#include "lib.h"

#define Min4(a,b) ((a)<(b))?(a):(b)

void main (int argc,char **argv)
{
  double x=2,y=3;
  printf("Min1(%f,%f)=%f\n",x,y,Min1(x,y));
  printf("Min3(%f,%f)=%f\n",x,y,Min3(x,y));
  printf("Min4(%f,%f)=%f\n",x,y,Min4(x,y));
  Min2(&x,y);
  printf("Min4=%f\n",x);
}
\end{verbatim}
Observe a funcionamento da linha \verb|return a<b?a:b;| na função \verb|Min3|. Essa é uma forma de escrever a condição
\begin{verbatim}
if (a<b) return a;
else return b;
\end{verbatim}
A expressão \verb|a<b?a:b| devolve o valor do teste: Se \verb|(a<b)| devolve \verb|a|, senão devolve \verb|b|. O macro \verb|#define Min4(a,b) ((a)<(b))?(a):(b)| também calcula o mínimo entre \verb|a| e \verb|b|. Um macro está escrito em uma única linha que inicia com \verb|#|. Se a expressão for muito grande e você precisar de duas linhas, use o operador continuidade \verb|\|:
\begin{verbatim}
#define Min4(a,b) ((a)<(b))?\
                        (a):\
                        (b)
\end{verbatim}

\section{Compilação condicional}
Quando não queremos compilar um pedaço de código, podemos usar a opção comentar \verb|//| ou \verb|/* */|. Mas as vezes, queremos mais de uma versão do código, compilando pedaços segundo uma condição que escolhemos. Usamos a sintaxe
\begin{verbatim}
#if condicao_1
  instrucao_1
#elif condicao_2 
  instrucao_2
#else 
  instrucao_padrao
#endif
\end{verbatim}
\begin{ex}
Faça um programa com dois \verb|main()|, um curto para fazer DEBUG e outro longo com um código que some $n$ números.
\end{ex}
\begin{verbatim}
#include <stdio.h>
#include <stdlib.h>

#define DEBUG 1

#if (DEBUG==1)
void main (int argc,char **argv)
{
  double soma=0;
  int i;
  for (i=1;i<argc;i++) soma+=atof(argv[i]);
  printf("A soma é =%f\n",soma);
}

#else
void main (int argc,char **argv)
{
  printf("Entrou com %d números\n ",argc-1);
}
#endif
\end{verbatim}


Também podemos usar as diretivas \verb|#ifdef|, \verb|#ifndef| e \verb|#undef| para escolher um pedaço de código na compilação. A diretiva \verb|#ifdef| verifica se um determinado símbolo está definido, \verb|#ifndef| verifica que um determinado símbolo não está definido e \verb|#undef| retira a definição de um símbolo.
\begin{ex}
Faça um programa que soma $n$ números de duas formas:
\begin{itemize}
 \item Se ligar a condição linha de comando, os números são lidos da linha de comando.
 \item senão, os números são lidos da tela com \verb|scanf|.
 \end{itemize}
\end{ex}
\begin{verbatim}
#include <stdio.h>
#include <stdlib.h>

//#define DEBUG 

#ifdef DEBUG
void main (int argc,char **argv)
{
#endif
#ifndef DEBUG
void main (void)
{
#endif
  double soma=0;
  int i;
#ifndef DEBUG
  int n;
  printf("Digite quantos números você vai somar\n");
  scanf("%d",&n);
  float v[n];
  printf("Digite os números\n");
  for (i=0;i<n;i++) scanf("%f",&v[i]);
  double argc=n+1;
#endif
#ifdef DEBUG
  float v[argc-1];
  for (i=0;i<argc-1;i++) v[i]=atof(argv[i+1]);
#endif
  for (i=0;i<argc-1;i++) soma+=v[i];
  printf("A soma é =%f\n",soma);
}
\end{verbatim}

O operador \verb|defined()| pode ser usado no teste da compilação condicional: \verb|#ifdef DEBUG| é o mesmo que \verb|#if defined(DEBUG)|. É possível trabalhar com mais de um símbolo, por exemplo, \verb|#if defined(DEBUG1) && !defined(DEBUG2)|, que compila quando \verb|DEBUG1| e \verb|DEBUG2| não está.

\section{assert()}
É uma macro da biblioteca \verb|assert.h| que entra um valor lógico, 0 ou 1. No caso de verdadeiro, não acontece nada. No caso falso, a função imprime o nome do arquivo fonte, a linha do arquivo contendo a chamada para a macro, o nome da função que contém a chamada e o texto da expressão que foi avaliada.
\begin{ex}
Implemente um código que lê $n$ números na linha de comando e imprime a soma. No caso de não entrar número, abortar a execução.
\end{ex}
\begin{verbatim}
#include <stdio.h>
#include <stdlib.h>
#include <assert.h>

void main (int argc,char **argv)
{
  assert(argc-1);
  double soma=0;
  int i;
  for (i=1;i<argc;i++) soma+=atof(argv[i]);
  printf("A soma é =%f\n",soma);
}
\end{verbatim}

\section{Exercícios}
\begin{exer}Calcule as raízes da função
$f(x)=x-\cos(x)$
usando os métodos de Newton e Secantes. Use \verb|#ifdef - #endif| para programar as duas opções. Veja os algorítmos nos exercícios \ref{exerc5.1} e \ref{exerc5.1.1}.
\end{exer}